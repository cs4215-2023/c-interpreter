\documentclass[a4paper]{article}

\usepackage[utf8]{inputenc}
\usepackage{amstext}
\usepackage{amsmath}
\usepackage{amssymb}
\usepackage{color,soul}
\usepackage{tikz}
\usepackage{listings, lstautogobble}
\usepackage{xcolor}
\usepackage{url}
\usepackage{hyperref}
\usepackage{multirow}

% For code listing %
\definecolor{codegreen}{rgb}{0,0.6,0}
\definecolor{codegray}{rgb}{0.5,0.5,0.5}
\definecolor{codepurple}{rgb}{0.58,0,0.82}
\definecolor{backcolour}{rgb}{0.95,0.95,0.92}

% Rule Commands
\newcommand{\Rule}[2]{\genfrac{}{}{0.7pt}{}{{\setlength{\fboxrule}{0pt}\setlength{\fboxsep}{3mm}\fbox{$#1$}}}{{\setlength{\fboxrule}{0pt}\setlength{\fboxsep}{3mm}\fbox{$#2$}}}}
\newcommand{\Rulee}[3]{\genfrac{}{}{0.7pt}{}{{\setlength{\fboxrule}{0pt}\setlength{\fboxsep}{3mm}\fbox{$#1$}}}{{\setlength{\fboxrule}{0pt}\setlength{\fboxsep}{3mm}\fbox{$#2$}}}[#3]}
\newcommand{\RuleWithName}[3]{\genfrac{}{}{0.7pt}{}{{\setlength{\fboxrule}{0pt}\setlength{\fboxsep}{3mm}\fbox{$#1$}}}{{\setlength{\fboxrule}{0pt}\setlength{\fboxsep}{3mm}\fbox{$#2$}}}[\text{#3}]}
\newcommand{\transition}{\rightrightarrows_s}
\newcommand{\translate}{\twoheadrightarrow}
\newcommand{\translateaux}{\hookrightarrow}
\newcommand{\Return}{\texttt{return}}
\newcommand{\Recfun}{\texttt{recfun}}
\newcommand{\If}{\texttt{if}}
\newcommand{\Then}{\texttt{then}}
\newcommand{\Else}{\texttt{else}}
\newcommand{\End}{\texttt{end}}
\newcommand{\Let}{\texttt{const}}
\newcommand{\In}{\texttt{in}}
\newcommand{\Rc}{\texttt{\}}}
\newcommand{\Lc}{\texttt{\{}}
\newcommand{\Times}{\texttt{*}}
\newcommand{\Plus}{\texttt{+}}
\newcommand{\TruE}{\texttt{true}}
\newcommand{\FalsE}{\texttt{false}}
\newcommand{\Int}{\texttt{int}}
\newcommand{\Float}{\texttt{float}}
\newcommand{\String}{\texttt{string}}
\newcommand{\Char}{\texttt{char}}
\newcommand{\Num}{\texttt{numeric}}
\newcommand{\Contract}{\texttt{contract}}
\newcommand{\Bool}{\texttt{bool}}
\newcommand{\Undefined}{\texttt{undefined}}
\newcommand{\X}{\texttt{x}}
\newcommand{\F}{\texttt{f}}
\newcommand{\LET}{\texttt{LET}}
\newcommand{\END}{\texttt{END}}
\newcommand{\IN}{\texttt{IN}}
\newcommand{\eval}{\rightarrowtail}
\newcommand{\evaL}{\rightarrowtail}
\newcommand{\partfun}{\rightsquigarrow}
\newcommand{\under}{\Vdash}
\newcommand{\ErroR}{\bot}
\newcommand{\Rp}{\texttt{)}}
\newcommand{\Lp}{\texttt{(}}
\newcommand{\Rb}{\texttt{\}}}
\newcommand{\Lb}{\texttt{\{}}
\newcommand{\blame}{\texttt{blame }}
\newcommand{\flatCon}{\texttt{con}}
\newcommand{\funCon}{\texttt{con1} \rightarrow \texttt{con2}}
\newcommand{\isFlat}{\texttt{flat}}
\newcommand{\conA}{\texttt{con1}}
\newcommand{\conB}{\texttt{con2}}

\lstdefinestyle{mystyle}{
    backgroundcolor=\color{backcolour},
    commentstyle=\color{codegreen},
    keywordstyle=\color{magenta},
    numberstyle=\tiny\color{codegray},
    stringstyle=\color{codepurple},
    basicstyle=\ttfamily\footnotesize,
    breakatwhitespace=false,
    breaklines=false,
    captionpos=b,
    keepspaces=false,
    showspaces=false,
    showstringspaces=false,
    showtabs=false,
    tabsize=2,
    autogobble=true
}

\lstset{style=mystyle}

\evensidemargin 35pt % Align even and odd numbered pages
\setlength{\parindent}{0in} % Paragraph Indentation
\setlength{\parskip}{\medskipamount} % Spaces between paragraphs

% Horizontal Margins
\setlength{\oddsidemargin}{-0.25in} % Left margin 1 inch (0 + 1)
\setlength{\textwidth}{6.75in} % Text width 6.5 inch (so right margin 1 inch).

% Vertical Margins
\setlength{\topmargin}{-0.75in} % Top margin 0.5 inch (-0.5 + 1)
\setlength{\headheight}{0.25in} % Head height 0.25 inch (where page headers go)
\setlength{\headsep}{0.25in} % Head separation 0.25 inch (between header and top line of text)
\setlength{\textheight}{10.25in} % Text height 8.5 inch (so bottom margin 1.5 in)

\title{A Realistic Explicit Control Evaluator for C}
\author{Lui Wen-Jie, Benjamin (A0214362N), Lester Leong }
\date{April 2022}

\begin{document}
\pagenumbering{gobble}% Remove page numbers (and reset to 1)

% Cover Page %
\begin{titlepage}
    \begin{center}
        \vspace*{1cm}

        \LARGE
        \textbf{A Realistic Explicit Control Evaluator for C}

        \vspace{0.5cm}        
		Project report for \textbf{CS4215}

        \vspace{3cm}

        \Large
        \textbf{Lester Leong (), \\Lui Wen-Jie, Benjamin (A0214362N)}\\
        School of Computing\\
        National University of Singapore\\
        April 2022

		\vspace{3cm}
		\small
		Frontend: \url{https://github.com/cs4215-2023/CS4215-frontend} \\
		Language Processor: \url{https://github.com/cs4215-2023/c-interpreter}

		\vfill

    \end{center}
\end{titlepage}

\tableofcontents
\newpage

\pagenumbering{arabic}
\section{Introduction}

% Outline of the scope and objectives of the project.
C is a programming language that has memory models present such as stack and heap. Thus, we decided to implement a realistic explicit control evaluator for C (RECEC) to align closely with the memory models present. \\

The first section, Section 2, will cover the features implemented. The second main section, Section 3, will include the technical specification of our RECEC.

\subsection{T-Diagrams}

T-diagrams of all major language processing steps that the project utilizes

\section{Features}

% Description of how the system can be built using the repository (or repositories) Description of test cases and how to run them: Make sure you include at least 10 test cases that cover the main features of the system. Clear specification of the envisioned deliverable (informal or formal)
% Description to what extent the objectives are met

This section will list the various features that were implemented and certain design choices that we made. The system can be built and tested by following the instructions of the language processor repository, followed by the frontend repository accordingly. Alternatively, one can use the dedicated code editor at %insert github pages link here
\\

Comments are currently \textbf{not supported}, but for the clarity of code examples, they will be use just like how it is implemented in C. Not that these examples are \textbf{not valid programs} in RECEC because of the comments present.  

\subsection{Primitive Type}
RECEC supports the basic types present in C, primarily, \texttt{int}, \texttt{void}, \texttt{float}, \texttt{char} and their respective pointer and array types.

\subsection{Basic Operators}
RECEC supports various basic operators.

\begin{itemize}
	\item Unary \texttt{!}, \texttt{\&}, \texttt{*}, \texttt{+} and \texttt{-}.
	\item Arithmetic operators \texttt{+}, \texttt{-}, \texttt{/}, \texttt{*}, \texttt{\%}
	\item Bitwise operators \texttt{\&}, \texttt{|}, \texttt{\^},  \texttt{>>}, \texttt{<<}
	\item Logical operators \texttt{\&\&}, \texttt{||}, \texttt{\^}
	\item Update operators \texttt{++}, \texttt{--}
\end{itemize}

This is an example of how basic operators are used in RECEC: 

\begin{verbatim}
	!(1); // 0
	!(0); // 1
	int a = 1;
	&a; // Gives address of variable a
	*a; // Dereference variable a
	a++; // Variable a contains the value of 2
\end{verbatim}

As we are not supporting \texttt{\_bool\_} type of C, logical operators and unary \texttt{!} will give an integer value of \texttt{1} if \texttt{true} and \texttt{0} otherwise. 

For equality checking, \texttt{==} and \texttt{!=}, we check if the values are the same, regardless of whether they have the same address or not. 

\subsection{Conditionals}

There are 2 types of conditionals in C, namely \texttt{if-else} statements and conditional expressions. Example conditionals can be written as so:

\begin{verbatim}
	int a = 1;
	// if-else statement
	if (a == 0) {
		// do something
	} else {
		// do something else
	}

	// conditional expressions
	a == 0 ? 2 : 3;
\end{verbatim}


\end{document}